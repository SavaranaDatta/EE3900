\documentclass[journal,12pt,twocolumn]{IEEEtran}

\usepackage{setspace}
\usepackage{gensymb}
\singlespacing
\usepackage[cmex10]{amsmath}

\usepackage{amsthm}
\usepackage{commath}
\usepackage{mathrsfs}
\usepackage{txfonts}
\usepackage{stfloats}
\usepackage{bm}
\usepackage{cite}
\usepackage{cases}
\usepackage{subfig}

\usepackage{longtable}
\usepackage{multirow}

\usepackage{enumitem}
\usepackage{mathtools}
\usepackage{steinmetz}
\usepackage{tikz}
\usepackage{circuitikz}
\usepackage{verbatim}
\usepackage{tfrupee}
\usepackage[breaklinks=true]{hyperref}
\usepackage{graphicx}
\usepackage{tkz-euclide}

\usetikzlibrary{calc,math}
\usepackage{listings}
    \usepackage{color}                                            %%
    \usepackage{array}                                            %%
    \usepackage{longtable}                                        %%
    \usepackage{calc}                                             %%
    \usepackage{multirow}                                         %%
    \usepackage{hhline}                                           %%
    \usepackage{ifthen}                                           %%
    \usepackage{lscape}     
\usepackage{multicol}
\usepackage{chngcntr}

\DeclareMathOperator*{\Res}{Res}

\renewcommand\thesection{\arabic{section}}
\renewcommand\thesubsection{\thesection.\arabic{subsection}}
\renewcommand\thesubsubsection{\thesubsection.\arabic{subsubsection}}

\renewcommand\thesectiondis{\arabic{section}}
\renewcommand\thesubsectiondis{\thesectiondis.\arabic{subsection}}
\renewcommand\thesubsubsectiondis{\thesubsectiondis.\arabic{subsubsection}}
\newtheorem{theorem}{Theorem}[section]
\newtheorem{corollary}{Corollary}[theorem]
\newtheorem{lemma}[theorem]{Lemma}
\newtheorem{definition}{Definition}[section]

\hyphenation{op-tical net-works semi-conduc-tor}
\def\inputGnumericTable{}                                 %%

\lstset{
%language=C,
frame=single, 
breaklines=true,
columns=fullflexible
}
\begin{document}

\newcommand{\BEQA}{\begin{eqnarray}}
\newcommand{\EEQA}{\end{eqnarray}}
\newcommand{\define}{\stackrel{\triangle}{=}}
\bibliographystyle{IEEEtran}
\raggedbottom
\setlength{\parindent}{0pt}
\providecommand{\mbf}{\mathbf}
\providecommand{\pr}[1]{\ensuremath{\Pr\left(#1\right)}}
\providecommand{\qfunc}[1]{\ensuremath{Q\left(#1\right)}}
\providecommand{\sbrak}[1]{\ensuremath{{}\left[#1\right]}}
\providecommand{\lsbrak}[1]{\ensuremath{{}\left[#1\right.}}
\providecommand{\rsbrak}[1]{\ensuremath{{}\left.#1\right]}}
\providecommand{\brak}[1]{\ensuremath{\left(#1\right)}}
\providecommand{\lbrak}[1]{\ensuremath{\left(#1\right.}}
\providecommand{\rbrak}[1]{\ensuremath{\left.#1\right)}}
\providecommand{\cbrak}[1]{\ensuremath{\left\{#1\right\}}}
\providecommand{\lcbrak}[1]{\ensuremath{\left\{#1\right.}}
\providecommand{\rcbrak}[1]{\ensuremath{\left.#1\right\}}}
\theoremstyle{remark}
\newtheorem{rem}{Remark}
\newcommand{\sgn}{\mathop{\mathrm{sgn}}}
\providecommand{\abs}[1]{\vert#1\vert}
\providecommand{\res}[1]{\Res\displaylimits_{#1}} 
\providecommand{\norm}[1]{\lVert#1\rVert}
%\providecommand{\norm}[1]{\lVert#1\rVert}
\providecommand{\mtx}[1]{\mathbf{#1}}
\providecommand{\mean}[1]{E[ #1 ]}
\providecommand{\fourier}{\overset{\mathcal{F}}{ \rightleftharpoons}}
%\providecommand{\hilbert}{\overset{\mathcal{H}}{ \rightleftharpoons}}
\providecommand{\system}{\overset{\mathcal{H}}{ \longleftrightarrow}}
	%\newcommand{\solution}[2]{\textbf{Solution:}{#1}}
\newcommand{\solution}{\noindent \textbf{Solution: }}
\newcommand{\cosec}{\,\text{cosec}\,}
\providecommand{\dec}[2]{\ensuremath{\overset{#1}{\underset{#2}{\gtrless}}}}
\newcommand{\myvec}[1]{\ensuremath{\begin{pmatrix}#1\end{pmatrix}}}
\newcommand{\mydet}[1]{\ensuremath{\begin{vmatrix}#1\end{vmatrix}}}
\numberwithin{equation}{subsection}
\makeatletter
\@addtoreset{figure}{problem}
\makeatother
\let\StandardTheFigure\thefigure
\let\vec\mathbf
\renewcommand{\thefigure}{\theproblem}
\def\putbox#1#2#3{\makebox[0in][l]{\makebox[#1][l]{}\raisebox{\baselineskip}[0in][0in]{\raisebox{#2}[0in][0in]{#3}}}}
     \def\rightbox#1{\makebox[0in][r]{#1}}
     \def\centbox#1{\makebox[0in]{#1}}
     \def\topbox#1{\raisebox{-\baselineskip}[0in][0in]{#1}}
     \def\midbox#1{\raisebox{-0.5\baselineskip}[0in][0in]{#1}}
\vspace{3cm}
\title{Assignment 5}
\author{Savarana Datta - AI20BTECH11008}
\maketitle
\newpage
\bigskip
\renewcommand{\thefigure}{\theenumi}
\renewcommand{\thetable}{\theenumi}
Download all python codes from 
\begin{lstlisting}
https://github.com/SavaranaDatta/EE3900/blob/main/EE3900_As5/codes/EE3900_As5.py
\end{lstlisting}
%
Download latex-tikz codes from 
%
\begin{lstlisting}
https://github.com/SavaranaDatta/EE3900/blob/main/EE3900_As5/EE3900_As5.tex
\end{lstlisting}
\section{Problem(Quadratic Forms Q.2.5)}
Find the area of the region in the first quadrant enclosed by x-axis, line $\myvec{1 & -\sqrt{3}}\vec{x}=0$ and the circle $\vec{x}^\top\vec{x}=4$.
\section{Solution}
\begin{lemma}
The points of intersection of line \\L:$\vec{x}=\vec{q}+\mu \vec{m}$ with the conic 
\begin{align}
\vec{x}^\top\vec{V}\vec{x}+2\vec{u}^\top\vec{x}+f=0
\end{align}
are given by 
\begin{align}  \vec{x_{i}}=\vec{q}+\mu_{i}\vec{m}
\end{align}
where 
\begin{multline}
    \mu_i=\frac{1}{\vec{m}^\top\vec{V}\vec{m}}-\vec{m}^\top\brak{\vec{V}\vec{q}+\vec{u}}\\ \pm\sqrt{[\vec{m}^\top(\vec{V}\vec{q}+\vec{u})]-(\vec{q}^\top\vec{V}\vec{q}+2\vec{u}^\top\vec{q}+\f)(\vec{m}^\top\vec{V}\vec{m})} 
\end{multline}

\end{lemma}
The equation of line $\myvec{1 & -\sqrt{3}}\vec{x}=0$ can also be expressed as $\vec{x}=\mu \myvec{1 \\ \frac{1}{\sqrt{3}}}$.\\
The matrix parameters of the circle $\vec{x}^\top\vec{x}=4$ are
\begin{align}
    \vec{V}&=\myvec{1&0\\0&1}\\
    \vec{u}&=\myvec{0\\0}\\
    f &= -4
\end{align}

The points of intersection of the line $\vec{x}=\mu \myvec{1 \\ \frac{1}{\sqrt{3}}}$ and the circle $\vec{x}^\top\vec{x}=4$ are 
\begin{align}
    \vec{x_i}&=\vec{q}+\mu_i\vec{m}\\
             &=\mu_i\myvec{1\\ \frac{1}{\sqrt{3}}}
\end{align}
where 
\begin{align}
    \mu_i = \pm\sqrt{3}
\end{align}
As the required point of intersection lies in first quadrant we have $\mu_i>0$
\begin{align}
    \implies \vec{A}=\sqrt{3}\myvec{1\\ \frac{1}{\sqrt{3}}}=\myvec{\sqrt{3}\\1}
\end{align}
The points of intersection of the line $\vec{x}=\mu\myvec{1\\0}$ and the circle $\vec{x}^\top\vec{x}=4$ are 
\begin{align}
    \vec{X_j}&=\vec{q}+\mu_j\vec{m}\\
             &=\mu_j\myvec{1\\0}
\end{align}
where
\begin{align}
    \mu_j = \pm 2
\end{align}
As the required point of intersection lies in first quadrant we have $\mu_j>0$
\begin{align}
    \implies \vec{B}&=2\myvec{1\\0}=\myvec{2\\0}
\end{align}
The angle($\theta$) of the sector AOB is 
\begin{align}
    \cos{\theta}&=\frac{\vec{A}^{\top}\vec{B}}{\norm{\vec{A}}\norm{\vec{B}}}\\
    &=\frac{2\sqrt{3}}{2\times 2}\\
    &=\frac{\sqrt{3}}{2}\\
    \implies \theta &= 30^{\degree}
\end{align}
\begin{align}
    \text{Area of the sector}&=\brak{\frac{\theta}{360^{\degree}}}\pi r^2\\
    &= \frac{\pi}{3}
\end{align}

\begin{figure}[!h]
 \centering
 \includegraphics[width=\columnwidth]{download.png}
 \caption{Reference plot}
 \label{plot}
\end{figure}
\end{document}
